% latex table generated in R 2.13.2 by xtable 1.6-0 package
% Fri Dec  7 21:09:35 2012
\begin{table}[ht]
\begin{center}
\begin{tabular}{rrrr}
  \hline
 & 1 & 2 & 3 \\ 
  \hline
1 & 0.01 & -0.16 & 0.18 \\ 
  2 & -0.03 & -0.16 & 0.10 \\ 
  3 & -0.03 & -0.16 & 0.11 \\ 
  4 & 0.05 & -0.06 & 0.16 \\ 
  5 & 0.04 & -0.07 & 0.15 \\ 
  6 & 0.03 & -0.09 & 0.15 \\ 
   \hline
\end{tabular}
\end{center}
\end{table}
